\section{Examples (performed with version 4.2)}


\subsection{An explosion in PREM}
\label{expl_PREM}

The first example I provide to illustrate the capability of \textit{RegSEM} is a simulation in PREM
\Citep{PREM} of a wavefield induced by a 10\,km deep explosion. The mesh used to perform this simulation is
shown in figure \ref{PREMmesh}. The elements of this mesh are such that a 20\,s period wavefield can be
propagated when using a polynomial order $N=4$. The waveforms obtained on the surface at $\mbox{20}^{\circ}$
epicentral distance are shown and compared to a normal mode solution in figure \ref{wavePREM}.\\
Inputs and outputs of a very similar simulation are in the directory \textit{\bfseries WORKDIR1}. Note that
module \textit{read\_model.f90} from the directory \textit{\bfseries SOLVE/src/Modules/PREM} has been
used. Note also that a smooth version of model CRUST2.0 \Citep {CRUST2} can be easily added on the top of
the PREM model using \textit{DATA/CRUST2.0/Moho.asc} in the mesher and \textit{DATA/CRUST2.0/Crust.asc}
when running the solver.

\begin{figure}
\vspace{-2cm}
\hspace{1cm}
\centerline{\includegraphics[angle=90,width=0.8\linewidth]{figures/waveformsPREM.pdf}}
\vspace{-1.5cm}
\caption{Comparison of the SEM solution (dashed black) with the normal mode
solution (red) obtained in the PREM with attenuation. The epicentral distance is
$\mbox{20}^{\circ}$. The residual multiplied by ten is plotted in green.}
\label{wavePREM}
\end{figure}


\subsection{An earthquake in CUB}
\label{expl_CUB}

The second example is a simulation in CUB \Citep{ShapRitz02} of a wavefield generated by an earthquake which
occured along the Mid-Atlantic Ridge on May 25, 2010. The mesh used to perform this simulation is shown in
figure \ref{CUBmesh}. Again, the elements of this mesh are such that a 20\,s period wavefield can be
propagated when using a polynomial order $N=4$. A map of the source-receiver configuration is shown in figure
\ref{CUBmoho}. The waveforms obtained at four stations are shown in figure \ref{waveCUB}. These waveforms are
compared with real data and synthetics computed in PREM and PREM + CRUST2.0.\\
Inputs and outputs of a very similar simulation are in the directory \textit{\bfseries WORKDIR2}. Note that
the \textit{Moho.asc} file from the directory \textit{\bfseries DATA/CUB} has been used to generate the mesh.
Note also that the \textit{read\_model.f90} module from the directory \textit{\bfseries SOLVE/src/Modules/CUB}
has been used in the solver. This module reads the file \textit{goodCUB.asc} which is in
\textit{\bfseries DATA/CUB}.

\begin{figure}
\centerline{\includegraphics[width=0.8\linewidth]{figures/CUBmoho.pdf}}
\caption{Map of the Moho of model CUB in the Atlantic-European region. The
source-receiver configuration used in subsection \ref{expl_CUB} is also shown.}
\label{CUBmoho}
\end{figure}

\begin{figure}
\hspace{0.5cm}
\centerline{\includegraphics[angle=90,width=\linewidth]{figures/waveform_atl.pdf}}
\caption{Waveforms induced by the Atlantic earthquake at four stations in Europe
for four different period bands. We compare the real data (blue) with the
waveforms obtained in three different Earth models (PREM in black, PREM+Crust2.0
in red and CUB in green).}
\label{waveCUB}
\end{figure}
